\newcommand{\departmentname}{Mathematics Department}
\newcommand{\coursename}{Linear Algebra}
\newcommand{\courseflavour}{Science}
\newcommand{\coursecode}{201-NYB-05}
\newcommand{\ponderation}{3-2-3}
\newcommand{\prereqs}{Good standing in High school or CEGEP Functions or equivalent.\newline
Note, however, that the majority of the students who take this course have already passed Calculus I and Calculus II, so they exhibit a fair degree of mathematical maturity.}
\newcommand{\textbook}{\emph{Elementary Linear Algebra (Abridged 10th edition)} by H. Anton.}
\newcommand{\references}{1. \emph{Linear Algebra with Applications} by W.K. Nicholson.\newline
2. \emph{Linear Algebra -- Ideas and Applications} by R.C. Penney.
}
\newcommand{\methodology}{Lectures and problem solving sessions.}
\newcommand{\gradingpolicy}{The final grade shall consist of the greater of:\\
				& $(A)$ Termwork for $50\%$ and Final Examination for $50\%$ of final grade\\
				& $(B)$ Termwork for $25\%$ and Final Examination for $75\%$ of final grade}
\newcommand{\objectives}{This course includes the study of systems of linear equations and elementary operations, matrices and determinants, vectors, lines, planes and vector spaces. For more details, see Dawson Science Program.}
