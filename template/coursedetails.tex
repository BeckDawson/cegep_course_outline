\newcommand{\departmentname}{Mathematics Department}
\newcommand{\coursename}{Calculus II}
\newcommand{\courseflavour}{Science}
\newcommand{\coursecode}{201-NYB-05}
\newcommand{\ponderation}{3-2-3}
\newcommand{\prereqs}{Calculus I \textit{(201-NYA/201-NYA-05)} preferably the Science version.}
\newcommand{\textbook}{\emph{Single Variable Calculus: Early Transcendentals (3rd edition)} by Jon Rogawski and Colin Adams.}
\newcommand{\references}{\emph{Single Variable Calculus: Early Transcendentals (8th edition)} by James Stewart, or any standard text book on calculus of a single variable.}
\newcommand{\methodology}{Lectures and problem solving sessions.}
\newcommand{\gradingpolicy}{The final grade shall consist of the greater of:\\
				& $(A)$ Termwork for $50\%$ and Final Examination for $50\%$ of final grade\\
				& $(B)$ Termwork for $25\%$ and Final Examination for $75\%$ of final grade}
\newcommand{\objectives}{This course introduces the student to Integral Calculus, to the techniques of integration and to some of the applications of integration to physical problems. Another look at limits and an introduction to the topic of infinite series are included.  For more details, see pages 44 to 49 of the Dawson Science Program.}
