\newcommand{\cheating}{
\large{\textbf{Policy on Cheating and Plagiarism}}\\
\vspace{-0.15in}\hline \bigskip \normalsize
\textbf{Cheating in Examinations, Tests, and Quizzes:}\\
Cheating includes any dishonest or deceptive practice relative to
formal final examinations, in-class tests, or quizzes. Such cheating
is discoverable during or after the exercise in the evaluation
process by the instructor. Such cheating includes, but is not limited
to\vspace{-0.1in}
\begin{enumerate}\addtolength{\itemsep}{-0.5\baselineskip}
\item[a.] copying or attempting to copy another's work.
\item[b.] obtaining or attempting to obtain unauthorized assistance of any kind.
\item[c.] providing or attempting to provide unauthorized assistance of any kind.
\item[d.] using or possessing any unauthorized material or instruments which can be used as information storage and retrieval devices.
\item[e.] taking an examination, test, or quiz for someone else.
\item[f.] having someone take an examination, test, or quiz in one's place. \smallskip
\end{enumerate}\vspace{-0.1in}

\textbf{Unauthorized Communication:}\\
Unauthorized communication of any kind during an examination,
test, or quiz is forbidden and subject to the same penalties as
cheating.\\\vspace{-0.1in}

\textbf{Plagiarism on Assignments and the Comprehensive Assessment:}\\
Plagiarism is the presentation or submission by a student of
another person's assignments or Comprehensive Assessment as his or
her own. Students who permit their work to be copied are considered
to be as guilty as the plagiarizer.\\\vspace{-0.1in}

\textbf{Obligation of the Teacher:}\\
Every instance of cheating or plagiarism leading to a resolution
that impacts on a student's grade must be reported by the teacher,
with explanation, in writing to the Chair of Mathematics and to the
Dean of Pre-University Studies. A copy of this report must also be
given to the student.\\\vspace{-0.1in}

\textbf{Penalties:}\\
Cheating and plagiarism are considered extremely serious academic
offences. Action in response to an incident of cheating and
plagiarism is within the authority of the teacher.
Penalties may range from zero on a test, to failure of the
course, to suspension or expulsion from the college.\\ \\ \vspace{-0.1in}

\textit{According to ISEP (Section IV-C), the teacher is required to report all cases of cheating and plagarism affecting a student's grade to the Sector Dean.}
}

\newcommand{\literacy}{
\large{\textbf{Literacy Policy}}\\
\vspace{-0.15in}\hline \bigskip \normalsize
Problem solving is an essential component of this course.  Students will be expected to analyze problems stated in words, to present their solutions logically and coherently, and to display their answers in a form corresponding to the statement of the problem, including appropriate units of measurement.  Marks will be deducted for work which is inadequate in their respects, even though the answers may be numerically correct.
}

\newcommand{\conflicts}{
\large\textbf{Intensive Course Conflicts and Policy on Religious Observances}\\
\vspace{-0.15in}\hline \bigskip \normalsize

Students who intend to observe religious holidays or who take intensive courses \textit{\textbf{must inform their
teachers in writing within the first two weeks of classes}}, as prescribed in ISEP.
A form for this purpose is available at the end of this document. \\

\textit{ISEP Statement on Intensive Courses (Appendix 1):}\\\vspace{-0.1in}

Students have an obligation to make arrangements in advance to meet the requirements of any classes missed as a result of taking an intensive course which forces them to be absent from their regularly scheduled course offerings.  Students who take intensive courses may be subject to penalties for missing laboratories or tests prescribed in the course outline. \\

\textit{ISEP Policy on Religious Observances (Section III-D):}
\\\vspace{-0.1in}

Students who wish to observe religious holidays must also inform each of their teachers in writing within
the first two weeks of each semester of their intent to observe the holiday so that alternative arrangements
convenient to both the student and the teacher can be made at the earliest opportunity. The written notice
must be given even when the exact date of the holiday is not known until later. Students who make such
arrangements will not be required to attend classes or take examinations on the designated days, nor be
penalized for their absence. \\\vspace{-0.1in}

It must be emphasized, however, that this College policy should not	 be	 interpreted	 to	 mean	 that	 a
student can receive credit for work not performed. It is the student's responsibility to fulfill the
requirements of the alternative arrangement.
}
