\newcommand{\objectives}{This course introduces the student to Integral Calculus, to the techniques of integration and to some of the applications of integration to physical problems. Another look at limits and an introduction to the topic of infinite series are included.  For more details, see pages 44 to 49 of the Dawson Science Program.}

\newcommand{\competencydescriptions}{This course will allow the student to fully achieve the competency
\textit{``{\textbf{00UP:} To apply the methods of integral calculus to the study of functions and problem solving''}}
and contributes to the partial achievement of the competency
\textit{``\textbf{00UU:} To apply what they have learned to one or more subjects in the sciences''}.}

\newcommand{\competencydetails}{
	\begin{multicols}{2}

		Elements of the competency \textbf{00UP}:
		\begin{compactenum}
			\item Determine the indefinite integral of a function.
			\item Calculate the limits of indeterminate forms.
			\item Calculate the definite integral and the improper integral of a function on an interval.
			\item Express concrete problems as differential equations and solve simple differential equations.
			\item Calculate volumes, areas and lengths and draw
			two- and three-dimensional representations.
			\item Analyze the convergence of infinite series. \\ \\ \\ \\ \\
		\end{compactenum}

		Performance criteria (\textbf{00UP}):
		\begin{compactitem}
			\item Adequate two- and three-dimensional representations of surfaces and solids of revolution
			\item Algebraic operations in conformity with rules
			\item Correct choice and application of rules and techniques of integration
			\item Accuracy of calculations
			\item Justification of steps in the solution
			\item Correct interpretation of results \item Use of appropriate terminology
		\end{compactitem}

		\vfill
		\columnbreak

		Elements of the competency \textbf{00UU}:
		\begin{compactenum}
			\item Recognize the contribution of more than one
			scientific discipline to certain situations.
			\item Apply the experimental method.
			\item Solve problems.
			\item Use data-processing technologies.
			\item Reason logically.
			\item Communicate effectively.
			\item Show evidence of independent learning in their choice of documentation or laboratory instruments.
			\item Work as members of a team.
			\item Make connections between science, technology and social progress. \\
		\end{compactenum}

		Performance criteria (\textbf{00UU}):
		\begin{compactitem}
			\item Use of an interdisciplinary approach
			\item Consistency and rigour in problem-solving, and justification of the approach used
			\item Observance of the experimental method and, where applicable, the experimental procedure
			\item Clarity and precision in oral and written communication
			\item Correct use of the appropriate data-processing technology
			\item Appropriate choice of documents or laboratory instruments
			\item Significant contribution to the team
			\item Appropriate connections between science, technology and social progress
		\end{compactitem}
	\end{multicols}
}
